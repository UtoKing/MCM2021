\documentclass{mcmthesis}
\mcmsetup{CTeX = false,   % 使用 CTeX 套装时,设置为 true
        tcn = 2123481, problem =A,
        sheet = true, titleinsheet = true, keywordsinsheet = true,
        titlepage = true, abstract = true}
\usepackage{newtxtext}%\usepackage{palatino}
\title{The XXX problem}
\author{Miao Fangran, Wang Xizhi, Wang Zhuo}
\date{\today}
\begin{document}
\begin{abstract}
Do not include the name of your school, advisor, or team members on this or any page.

Papers must be within the 25 page limit.

Be sure to change the control number and problem choice above.
You may delete these instructions as you begin to type your report here.

Follow us @COMAPMath on Twitter or COMAPCHINAOFFICIAL on Weibo for the most up to date contest information.

\begin{keywords}
Python; Mathematical modelling.
\end{keywords}
\end{abstract}
\maketitle
%% Generate the Table of Contents, if it's needed.
\tableofcontents
\newpage
%%
%% Generate the Memorandum, if it's needed.
% \memoto{\LaTeX{}studio}
% \memofrom{Liam Huang}
% \memosubject{Happy \TeX{}ing!}
% \memodate{\today}
% % \logo{\LARGE I'm pretending to be a LOGO!}
% \begin{memo}[Memorandum]
% \end{memo}
%%
\section{Introduction}
\subsection{Problem Background}

\subsection{Literature Review}

\subsection{Our work}



\section{Preparation of the Models}
\subsection{Assumptions}
Here are our assumptions:

For a single species of fungus\cite{Lustenhouwer11551}:
\begin{itemize}
  \item Under an ideal environment, which means that the temperature is perfect and the food is sufficient, the decomposition rate of fungi is mainly related to the extension rate and the moisture trade-off.
  \item The decomposition rate has a linear relationship with extension rate.
  \item The logarithm of decomposition rate has a linear relationship with moisture trade-off.
  \item If the environment is not so perfect, we introduce one parameter to adjust the decomposition rate---temperature.
\end{itemize}

With such condition, we can conclude two partial differential equations:
\begin{equation}
  \label{eq6}
  \left\{
  \begin{aligned}
  &\frac{\partial f}{\partial x}&=C_1\\
  &\frac{\partial (\log f)}{\partial y}&=C_2
  \end{aligned}
  \right.
\end{equation}

Where $f$ represents the decomposition rate, $x$ represents extension rate and $y$ represents the moisture trade-off.
$C_1,C_2$ represents the linear slope between the independent variables and dependent variables.

Solve the equation, we acquire that:
\begin{equation}
  f=Ax\cdot e^{By}
\end{equation}
where $A$ and $B$ are constants, which varies with the type of fungi.

To adjust the decomposition rate with temperature, we introduce the Temperature function, which is $u(T)$.
Thus:
\begin{equation}
  f=u(T)\cdot Ax\cdot e^{By}
\end{equation}

When it comes to multiple fungi, we put forward our fifth assumption:
\begin{itemize}
  \item The decomposition rate will change when the proportion of fungi changes.
\end{itemize}
Then, we also introduce another function of one specific fungus' proportion:
\begin{equation}
  f=v(P)\cdot u(T)\cdot Ax\cdot e^{By}
\end{equation}

\section{The Models}

\section{Strengths and weaknesses}

\subsection{Strengths}


% \begin{thebibliography}{99}
% \bibitem{1} D.~E. KNUTH   The \TeX{}book  the American
% Mathematical Society and Addison-Wesley
% Publishing Company , 1984-1986.
% \bibitem{2}Lamport, Leslie,  \LaTeX{}: `` A Document Preparation System '',
% Addison-Wesley Publishing Company, 1986.
% \bibitem{3}\url{https://www.latexstudio.net/}
% \end{thebibliography}
\medskip
\bibliographystyle{unsrt}
\bibliography{references}

\begin{appendices}

\section{First appendix}

\section{Second appendix}

% some more text \textcolor[rgb]{0.98,0.00,0.00}{\textbf{Input C++ source:}}
% \lstinputlisting[language=C++]{./code/mcmthesis-sudoku.cpp}
\lstinputlisting[language=python]{./code/Fungi_2.py}
\end{appendices}
\end{document}
