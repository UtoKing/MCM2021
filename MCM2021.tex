\documentclass{mcmthesis}
\mcmsetup{CTeX = false,   % 使用 CTeX 套装时,设置为 true
        tcn = 2123481, problem =A,
        sheet = true, titleinsheet = true, keywordsinsheet = true,
        titlepage = true, abstract = true}
\usepackage{newtxtext}%\usepackage{palatino}
\title{The XXX problem}
\author{Miao Fangran, Wang Xizhi, Wang Zhuo}
\date{\today}
\begin{document}
\begin{abstract}
Do not include the name of your school, advisor, or team members on this or any page.

Papers must be within the 25 page limit.

Be sure to change the control number and problem choice above.
You may delete these instructions as you begin to type your report here.

Follow us @COMAPMath on Twitter or COMAPCHINAOFFICIAL on Weibo for the most up to date contest information.

\begin{keywords}
Matlab; Mathematical modelling.
\end{keywords}
\end{abstract}
\maketitle
%% Generate the Table of Contents, if it's needed.
\tableofcontents
\newpage
%%
%% Generate the Memorandum, if it's needed.
% \memoto{\LaTeX{}studio}
% \memofrom{Liam Huang}
% \memosubject{Happy \TeX{}ing!}
% \memodate{\today}
% % \logo{\LARGE I'm pretending to be a LOGO!}
% \begin{memo}[Memorandum]
% \end{memo}
%%
\section{Introduction}
\subsection{Problem Background}

\subsection{Literature Review}

\subsection{Our work}



\section{Preparation of the Models}
\subsection{Assumptions}
Here are our assumptions:
\begin{itemize}
  \item Under an ideal environment, which means that the temperature is perfect and the food is sufficient, the decomposition rate of fungis is mainly related to the extension rate and the moisture trade-off.
  \item The decomposition rate has a linear relationship with extension rate.
  \item The logarithm of decomposition rate has a linear relationship with moisture trade-off.
  \item If the environment is not so perfect, we introduce two parameters to adjust the decomposition rate---temperature and fungi's proportion.
\end{itemize}

With such condition, we can conclude two partial differential equations:


\section{The Models}

\section{Strengths and weaknesses}

\subsection{Strengths}
\cite{bush2006mathematical}


% \begin{thebibliography}{99}
% \bibitem{1} D.~E. KNUTH   The \TeX{}book  the American
% Mathematical Society and Addison-Wesley
% Publishing Company , 1984-1986.
% \bibitem{2}Lamport, Leslie,  \LaTeX{}: `` A Document Preparation System '',
% Addison-Wesley Publishing Company, 1986.
% \bibitem{3}\url{https://www.latexstudio.net/}
% \end{thebibliography}
\medskip
\bibliographystyle{unsrt}
\bibliography{references}

\begin{appendices}

\section{First appendix}

Here are simulation programmes we used in our model as follow.\\

\textbf{\textcolor[rgb]{0.98,0.00,0.00}{Input matlab source:}}
\lstinputlisting[language=Matlab]{./code/mcmthesis-matlab1.m}

\section{Second appendix}

some more text \textcolor[rgb]{0.98,0.00,0.00}{\textbf{Input C++ source:}}
\lstinputlisting[language=C++]{./code/mcmthesis-sudoku.cpp}

\end{appendices}
\end{document}
